\documentclass[letterpaper]{article}

\usepackage{spellcard}
\usepackage{ifthen}
\usepackage[fixed]{fontawesome5}

% colors
\definecolor{titlebg}{RGB}{30,30,30}
\definecolor{bannerbg}{RGB}{255,000,50}
\definecolor{textbg}{RGB}{200,200,200}

% card layout template
\newcommand{\card}{
    \cardtitle{titlebg}
    \cardbanner{bannerbg}
    \cardtextarea{textbg}
    \cardborder
    %\carddebug
}

% familiar card layout template
\newcommand{\familiar}{
    \cardtitle{titlebg}
    \cardbanner{bannerbg}
    \cardborder
    %\carddebug
}

% spell cards
\begin{document}
\begin{center}
    \pagestyle{empty}

    %% spell template
    %\begin{tikzpicture}
    %    \card
    %    \spellName{name}
    %    \spellIcon{school}
    %    \spellSchoolLevel{banner text}
    %    \spellComponents{V}{S}{M}
    %    \spellMaterialComponents{optional material components}
    %    \spellCastingTime{1 Action}
    %    \spellRitual %optional
    %    \spellRange{60 ft}
    %    \spellArea{15 ft}{cube} %optional
    %    \spellDuration{Instantaneous}
    %    \spellConcentration %optional
    %    \spellText{
    %        spell text, may contain TeX formatting\\
    %        \vrule width \textwidth height 0.5pt\\[3pt] %optional separator
    %        additional text such as up-level casting information
    %    }
    %\end{tikzpicture}

    %absorb elements
    \begin{tikzpicture}
        \card
        \spellName{Absorb Elements}
        \spellIcon{Abjuration}
        \spellSchoolLevel{1st Level Abjuration}
        \spellComponents{}{S}{}
        \spellCastingTime{1 Reaction}
        \spellRange{Self}
        \spellDuration{1 Round}
        \spellText{
            \scriptsize{\textit{Reaction: when you take acid, cold, fire,
            lightning, or thunder damage}}\\
            \vspace{3pt}
            The spell captures some of the incoming energy, lessening its effect
            on you and storing it for your next melee attack. You have
            \textbf{resistance to the triggering damage type} until the start of
            your next turn. Also, the first time you hit with a melee attack on
            your next turn, the target takes an \textbf{extra 1d6 damage} of the
            triggering type, and the spell ends.\\
            \vrule width \textwidth height 0.5pt\\[3pt]
            When you cast this spell using a spell slot of 2nd level or higher,
            the extra damage increases by \textbf{1d6} for each slot level above 1st.
        }
    \end{tikzpicture}
    \hspace{2mm}
    %alarm
    \begin{tikzpicture}
        \card
        \spellName{Alarm}
        \spellIcon{Abjuration}
        \spellSchoolLevel{1st Level Abjuration}
        \spellComponents{V}{S}{M}
        \spellMaterialComponents{a tiny bell and a piece of fine silver wire}
        \spellCastingTime{1 Minute}
        \spellRitual %optional
        \spellRange{30 ft.}
        \spellDuration{8 Hours}
        \spellText{
            \scriptsize{You set an alarm against unwanted intrusion. Choose a door, a
            window, or an area within range that is no larger than a
            \textbf{20-foot cube}. Until the spell ends, an alarm alerts you
            whenever a Tiny or larger creature touches or enters the warded area.
            When you cast the spell, you can designate creatures that won't set
            off the alarm. You also choose whether the alarm is mental or audible.\\
            \vspace{2mm}
            A mental alarm alerts you with a ping in your mind if you are within
            \textbf{1 mile} of the warded area. This ping awakens you if you are sleeping.\\
            \vspace{2mm}
            An audible alarm produces the sound of a hand bell for 10 seconds within 60 feet.}
        }
    \end{tikzpicture}
    \hspace{2mm}
    %absorb elements
    \begin{tikzpicture}
        \card
        \spellName{Shield}
        \spellIcon{Abjuration}
        \spellSchoolLevel{1st Level Abjuration}
        \spellComponents{V}{S}{}
        \spellCastingTime{1 Reaction}
        \spellRange{Self}
        \spellDuration{1 Round}
        \spellText{
            \scriptsize{\textit{Reaction: when you are hit by an attack or
            targeted by the magic missile spell}}\\
            \vspace{3pt}
            An invisible barrier of magical force appears and protects you.
            Until the start of your next turn, you have a
            \textbf{\texttt{+}5 bonus to AC}, including against the triggering
            attack, and you take no damage from magic missile.
        }
    \end{tikzpicture}
    %familiar
    \begin{tikzpicture}
        \familiar
        \spellName{Poot}
        \spellSchoolLevel{Owl Familiar}
        \node[right, text width=(1cm)] at (1.65,6) {
            \includegraphics[width=3.5cm, height=3.5cm]{img/owl.jpg}
        };
        \spellText{
            \vspace{20mm}
            \begin{center}
                \tiny{Tiny Fey, Unaligned}\\
                \vspace{2mm}
                \scriptsize
                \faIcon{shield-alt} 11 \hspace{2pt} \faIcon{heart}1 \hspace{2pt} \faIcon{shoe-prints} 5 ft. \faIcon{paper-plane} 60 ft.\\
                \vrule width \textwidth height 0.5pt\\[3pt]
                \begin{tabular}{ c c c }
                    \textbf{STR} & \textbf{DEX} & \textbf{CON}\\
                    3 (\texttt{-}4) & 13 (\texttt{+}1) & 8 (\texttt{-}1)\\
                    \\
                    \textbf{INT} & \textbf{WIS} & \textbf{CHA} \\
                    2 (\texttt{-}4) & 12 (\texttt{+}1) & 7 (\texttt{-}2)
                \end{tabular}\\
            \vrule width \textwidth height 0.5pt\\[3pt]
            \tiny
            \textbf{Skills} Perception \texttt{+}3, Stealth \texttt{+}3\\
            \vspace{1mm}
            \faIcon{eye} Darkvision 120 ft., Passive Perception 13\\
            \vspace{1mm}
            \textbf{Flyby.} Does not provoke opportunity attacks\\
            \vspace{1mm}
            \textbf{Keen Hearing and Sight.} Advantage on Perception checks that use hearing/sight
            \end{center}
        }
    \end{tikzpicture}
\end{center}
\end{document}
