\documentclass[letterpaper]{article}

\usepackage{spellcard}
\usepackage{ifthen}
\usepackage[fixed]{fontawesome5}

% colors
\definecolor{titlebg}{RGB}{30,30,30}
\definecolor{bannerbg}{RGB}{255,000,50}
\definecolor{textbg}{RGB}{200,200,200}

% card layout template
\newcommand{\card}{
    \cardtitle{titlebg}
    \cardbanner{bannerbg}
    \cardtextarea{textbg}
    \cardborder
    %\carddebug
}

% familiar card layout template
\newcommand{\familiar}{
    \cardtitle{titlebg}
    \cardbanner{bannerbg}
    \cardborder
    %\carddebug
}

% spell cards
\begin{document}
\begin{center}
    \pagestyle{empty}
    %% spell template
    %\begin{tikzpicture}
    %    \card
    %    \spellName{name}
    %    \spellIcon{school}
    %    \spellSchoolLevel{banner text}
    %    \spellComponents{V}{S}{M}
    %    \spellMaterialComponents{optional material components}
    %    \spellCastingTime{1 Action}
    %    \spellRitual %optional
    %    \spellRange{60 ft}
    %    \spellArea{15 ft}{cube} %optional
    %    \spellDuration{Instantaneous}
    %    \spellConcentration %optional
    %    \spellText{
    %        spell text, may contain TeX formatting\\
    %        \vrule width \textwidth height 0.5pt\\[3pt] %optional separator
    %        additional text such as up-level casting information
    %    }
    %\end{tikzpicture}

    % identify
    \begin{tikzpicture}
        \card
        \spellName{Misty Step}
        \spellIcon{Conjuration}
        \spellSchoolLevel{2nd Level Conjuration}
        \spellComponents{V}{}{}
        \spellCastingTime{1 Bonus Action}
        \spellRange{Self}
        \spellDuration{Instantaneous}
        \spellText{
            Briefly surrounded by silvery mist, you teleport up to
            \textbf{30 feet} to an unoccupied space that you can see.
        }
    \end{tikzpicture}
    % shatter
    \begin{tikzpicture}
        \card
        \spellName{Shatter}
        \spellIcon{Evocation}
        \spellSchoolLevel{2nd Level Evocation}
        \spellComponents{V}{S}{M}
        \spellMaterialComponents{a chip of mica}
        \spellCastingTime{1 Action}
        \spellRange{60 ft}
        \spellArea{10 ft}{sphere}
        \spellDuration{Instantaneous}
        \spellText{
            A sudden loud ringing noise, painfully intense, erupts from a point
            of your choice within range. Each creature in a \textbf{10-foot-radius sphere}
            centered on that point must make a \textbf{Constitution saving throw}.
            A creature takes \textbf{3d8 thunder damage} on a failed save,
            \textbf{or half} as much damage on a successful one. A creature made
            of inorganic material such as stone, crystal, or metal has
            \textbf{disadvantage on this saving throw}.\\
            \vrule width \textwidth height 0.5pt\\[3pt]
            When you cast this spell using a spell slot of 3rd level or higher,
            the damage increases by \textbf{1d8} for each slot level above 2nd.
        }
    \end{tikzpicture}
    % catapult
    \begin{tikzpicture}
        \card
        \spellName{Catapult}
        \spellIcon{Transmutation}
        \spellSchoolLevel{1st Level Transmutation}
        \spellComponents{}{S}{}
        \spellCastingTime{1 Action}
        \spellRange{60 ft}
        \spellDuration{Instantaneous}
        \spellText{
            Choose a \textbf{1-5lb} object within range not being worn or carried.
            It flies in a straight line up to \textbf{90 feet} before falling to
            the ground unless it impacts a solid surface. If it strikes a creature,
            that creature makes a \textbf{Dexterity saving throw} to avoid the hit.
            When the object strikes, it and what it strikes each take
            \textbf{3d8 bludgeoning damage}.
            \vrule width \textwidth height 0.5pt\\[3pt]
            \textbf{At Higher Levels.} When you cast this spell using a spell
            slot of 2nd level or higher, the maximum weight of the object increases
            by \textbf{5lbs}, and the \textbf{damage increases by 1d8}, for each
            slot level above 1st.
        }
    \end{tikzpicture}
\end{center}
\end{document}
\end{center}
\end{document}
